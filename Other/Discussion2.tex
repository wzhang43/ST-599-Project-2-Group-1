\documentclass[12pt]{article}

\begin{document}
\title{Discussion 2}
\maketitle

Problems and Solutions 

Solution - by Matt

\begin{itemize}
\item Population-based approach  

we need data by :
Month, Airport, with proportion or mean of delay times,

SQL can do the group by and summarizing for us

we would run the same thing again for the arrival airport. We could modify ours to be specific about origin or destination, if necessary. That would give us a dataset that we could manipulate on our end, adding a "Region Code" column and then using that and dplyr to get at what we want. 

\item Population-based approach  

SQL has an "IN" function, just like R does
which( dataframe \%in\% c("fie", "fi", "fo", "fum")) returns the indexes from "dataframe" that have a value in the list of strings. 

we can make a list object where each item is a group of airport codes like so:

regionlist <- list(  c("HOU", "DFW"),c("PDX", "TAC", "SEA"),c("LAX", "SDI"),etc)

Then in our loop:

for( i in 1:whatever) {
    ...
    filter( origin \%in\% regionlist[i] )
    ...
}

\end{itemize}
\end{document}