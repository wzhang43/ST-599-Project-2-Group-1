\documentclass[serif,mathserif]{beamer}

\mode<presentation>
{
\usetheme{Frankfurt}
\setbeamercovered{transparent}
}


%\usepackage{beamerarticle}
\usepackage{amssymb}

\usepackage[english]{babel}
% or whatever

\usepackage[latin1]{inputenc}
% or whatever

\usepackage{times}
\usepackage[T1]{fontenc}
% Or whatever. Note that the encoding and the font should match. If T1
% does not look nice, try deleting the line with the fontenc.

\usepackage{amsfonts,amsmath, amsthm, amssymb, latexsym, epsfig}

\usepackage[all]{xy}
\input xy
\xyoption{all}
\usepackage{xspace}

\usepackage[mathscr]{eucal}

%%%%%%%%%%%%%%%%%%%%%%
\usepackage{hyperref}
\usepackage{graphicx}
\usepackage{multicol}
%%%%%%%%%%%%%%%%%%%%%%
%\usepackage{xcolor}


\newtheorem{conjecture}{Conjecture}
\newtheorem{proposition}{Proposition}
\newtheorem{remark}{Remark}
\newtheorem{question}{Question}

\title % (optional, use only with long paper titles)
{ST 599 Project 1}
%\subtitle{}

\author[] % (optional, use only with lots of authors)
{Wanli Zhang $\&$ Matt Edwards}
%{Chiaochih Chang \\ Dr. Ferry Butar Butar}
% - Give the names in the same order as the appear in the paper.
% - Use the \inst{?} command only if the authors have different
%   affiliation.

\institute[] % (optional, but mostly needed)
{Oregon State University}
  %\and
  %\inst{2}%
  %Department of Statistics \\
  %Piedmont College
% - Use the \inst command only if there are several affiliations.
% - Keep it simple, no one is interested in your street address.

\date{April 23rd, 2014} % (optional, should be abbreviation of conference name)
%{North Georgia College and State University\\ }
% - Either use conference name or its abbreviation.
% - Not really informative to the audience, more for people (including
%   yourself) who are reading the slides online



% If you have a file called "university-logo-filename.xxx", where xxx
% is a graphic format that can be processed by latex or pdflatex,
% resp., then you can add a logo as follows:

% \pgfdeclareimage[height=0.5cm]{university-logo}{university-logo-filename}
% \logo{\pgfuseimage{university-logo}}



% If you wish to uncover everything in a step-wise fashion, uncomment
% the following command:

%%%%%%%%%%%%%%%%%%%%%%%%%%%%%%%%%%%%%%\beamerdefaultoverlayspecification{<+->}

\newtheorem{myDef}{Definition}

\title{Analysis of geographical patterns in weather-related flight delays}

\author{Mathew Edwards, Nandhita Narendra Babu, Wanli Zhang}

\date{\today}






\begin{document}

\begin{frame}
\titlepage
\end{frame}

\begin{frame}{\contentsname}
\tableofcontents
\end{frame}

\section{Introduction}
\subsection{}

\begin{frame}
\frametitle{Data Information}
\begin{itemize}
\item On time flight data from the Bureau of Transportation Statistics for all US States
\item Weather-related flight delays data from June 2003 to December 2013
\end{itemize}
\end{frame}

\begin{frame}
\frametitle{Question of Interest}
Are there geographical patterns in weather-related flight delays, and do these change over time?
\end{frame}




#---------------------------------------------------------------------------------------------------#




\section{Population-based Analysis}
\begin{frame}
\frametitle{Population Based}
\begin{center} 

\end{center}
\end{frame}




#----------------------------------------------------------------------------------------------------#





\section{Sampling-based Analysis}
\begin{frame}
\frametitle{Sampling Technique}
\begin{center} 
\end{center}
\end{frame}





#-----------------------------------------------------------------------------------------------------#






\section{Discussion, Obstacles and Solutions}
\begin{frame}
\frametitle{Obstacles and Solutions}

\end{frame}





\end{document}
